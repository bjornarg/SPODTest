\section{Configuration}
\gls{spodtest} uses the configuration file \verb@spodconf.cfg@, which is an INI
formatted configuration file. The configuration file defines what tests are
run.

The tests are defined by the \verb@tests@ parameter in the \verb@spod@ section.
The value of this key should be a comma separated list of other sections in the
same configuration file. Each comma separated value represents a \gls{testset}.


\subsection{The test sections}
\label{sec:test_sec}


\paragraph*{Required options}

\begin{description}
    \item[type] The type of test to be run, this can be either: \verb@scp@,
        \verb@sftp@ or \verb@rsync@.
    \item[host] The host the test should run against.
    \item[arguments] A comma separated list of sections that represents a set
        of arguments to be run on the command. See section
        \ref{sec:argument_sec}.
\end{description}

\paragraph*{Optional options}

\begin{description}
    \item[username] The username used when connecting to the remote site. If
        this argument is not used, no username is used in the function and it
        will \textit{probably} default to using the same username as the one
        running the script.
\end{description}


\subsection{The argument sections}
\label{sec:argument_sec}

\paragraph*{Optional options}

\begin{description}
    \item[encryption]   Sets the encryption type used when transmitting data.
        Usually \verb@3des@ or \verb@blowfish@ are valid arguments.
    \item[compression]  Whether or not compression should be used when
        transmitting data. Valid values are \verb@yes@ and \verb@no@.
    \item[compression\_level]   Sets the level of compression used when
        transmitting the files. Allowed values are integers 1-9.
\end{description}
