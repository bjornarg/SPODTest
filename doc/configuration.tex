\section{Configuration}
\gls{spodtest} uses the configuration file \verb@spodconf.cfg@, which is an INI
formatted configuration file. The configuration file defines what tests are
run.

\subsection{The spod section}
This is the main section that defines where to find test setup information. 

\paragraph*{Required options}

\begin{description}
    \item[tests] The tests that should be run when \verb@spodtest.py@ is
        executed. This should be a comma separated list of other sections in
        the configuration file, with the keys described in section
        \ref{sec:test_sec}.
\end{description}

\paragraph*{Optional options}

\begin{description}
    \item[loglevel] The logging level for the script. This can be
        \textit{debug, info, warning, error} or \textit{critical}. Defaults to
        \textit{warning}. Recommended setting is \textit{warning}.
    \item[logfile] The log file to log to. This string will be formatted with
        the Python
        \href{http://docs.python.org/library/datetime.html#strftime-and-strptime-behavior}{datetime module}. 
        Defaults to \textit{spodtest.\%Y-\%m-\%d.log}.
\end{description}


\subsection{The test sections}
\label{sec:test_sec}

These sections will define each separate test.

\paragraph*{Required options}

\begin{description}
    \item[type] The type of test to be run, this can be either: \verb@scp@,
        \verb@sftp@ or \verb@rsync@.
    \item[host] The host the test should run against.
    \item[arguments] A comma separated list of sections that represents a set
        of arguments to be run on the command. See section
        \ref{sec:argument_sec}.
    \item[target\_folder] The folder to transfer files to on \textbf{host}.
    \item[files] The files to transfer.
\end{description}

\paragraph*{Optional options}

\begin{description}
    \item[username] The username used when connecting to the remote site. If
        this argument is not used, no username is used in the function and it
        will \textit{probably} default to using the same username as the one
        running the script.
\end{description}


\subsection{The argument sections}
\label{sec:argument_sec}

There are no required options for argument sections. If no options are given,
it will simply use the default values for each.

\paragraph*{Optional options}

\begin{description}
    \item[encryption]   Sets the encryption type used when transmitting data.
        Usually \verb@3des@ or \verb@blowfish@ are valid arguments.
    \item[compression]  Whether or not compression should be used when
        transmitting data. Valid values are \verb@yes@ and \verb@no@.
    \item[compression\_level]   Sets the level of compression used when
        transmitting the files. Allowed values are integers 1-9.
\end{description}

\newpage
\subsection{Example}

Below is an example of a \verb@spodconf.cfg@ file. This is set up for
transferring data to \verb@titan.uio.no@. It will test the commands
\verb@rsync@ and \verb@scp@ on the \verb@smalldata@ dataset. It will run one
test with compression enabled, and one with compression enabled and blowfish
encryption for each of the commands. 

It will log to the file \verb@/var/log/spod.YYYY-MM-DD.log@ with a log level of
\verb@warning@.

\begin{verbatim}
[spod]
logfile=/var/log/spod.%Y-%m-%d.log
loglevel=warning
tests=test1,test2

[test1]
type=rsync
arguments=args1,args2
host=titan.uio.no
target_folder=/home/test/datadump
files=/home/test/datasets/smalldata

[test2]
type=scp
arguments=args1,args2
host=titan.uio.no
target_folder=/home/test/datadump
files=/home/test/datasets/smalldata

[args1]
compression=yes

[args2]
encryption=blowfish
compression=yes

\end{verbatim}
