\section{Data sets}
The data sets should be a folder containing the data to be transfered. The
client will automatically calculate the size of the directory and the number of
files present. 

Data sets should \textit{should} be named. This is done by placing a text file
in the root of the data set folder named \verb@fs_name@. This should contain
only the name of the file set.

The \verb@fs_name@ file will not be counted in the number of files for the data
set. It \textit{will} however be sent, creating a slight inconsistency between
the data presented and the data transferred. This is due to \verb@scp@ not
being able to exclude certain files from transfer, so the data set could not be
properly transferred through a single command using \verb@scp@, while dropping
\verb@fs_name@ from the transferred files. Since most data sets are likely to
either contain a large number of files, or large files, the extra data sent
due to \verb@fs_name@ is considered to be negligible.
