\section{Extending SPODTest}

\gls{spodtest} can be extended to include tests for more commands if required.
This is done by subclassing the \verb@testers.base.CommandBuilder@ class. These
subclasses define the command itself, how it handles arguments, and can build
any necessary extra files needed for sending the data.

See \verb@testers.sftp@ for a well documented example of how to create such a
class.

\subsection{Class variables}

There are certain variables that should be set when a \verb@CommandBuilder@ sub
class is created. These are:

\begin{description}
    \item[base\_cmd] A string of the command to be executed.
    \item[test\_args] A list of \verb@Args@ class instances representing each
        test case for the command.
    \item[common\_args] A list of arguments (strings) that should be passed to
        every run of the command. \textit{This can be parameters that sets the
        program to recursive mode, batch mode or similar that are necessary for
        the program to execute properly without requiring input.}
    \item[cmd\_format] A string that represents how the command should be
        built. This string will be
        \href{http://docs.python.org/library/stdtypes.html#string-formatting}{string
        formatted} with a dictionary containing the values described in section
        \ref{sec:format_dict}.
\end{description}

\subsection{Format dictionary}
\label{sec:format_dict}

The format dictionary will contain the following key/value pairs:

\begin{description}
    \item[base\_cmd] The variable \verb@self.base_cmd@ set on creation.
    \item[common\_args] A concatenation of the strings set in
        \verb@self.common_args@.
    \item[test\_args] A concatenation of the string values set in each of the
        \verb@test_args@.
    \item[target] Target is either the host, or a combination of host and
        user name (if username is set in the test section) as
        \verb"username@host".
    \item[target\_folder] The \verb@target_folder@ set in the test section in
        the configuration file.
    \item[filename] The name of the file (empty string if the file is a
        directory).
    \item[filedir] Directory the file is in. Equivalent to
        \textbf{filelocation}  if the file is a directory.
    \item[filelocation] Absolute path to the file. Equivalent to
        \textbf{filedir} if the file is a directory.
    \item[usagefile] If any file is set as the usage file through
        \verb@FileObject.set_usage_file()@, this will be the variable. See
        \verb@testers.sftp@ for an example where usage file is used.
\end{description}
